\documentclass[12pt]{article}
\usepackage{geometry}
\geometry{a4paper, margin=1in}
\usepackage{graphicx} % Required for inserting images
\graphicspath{ {./figures/} }
\usepackage[utf8]{inputenc}
\usepackage{titling}
\usepackage{subfiles}

\newcommand{\subtitle}[1]{%
	\posttitle{%
		\par\end{center}
		\begin{center}\Large#1\end{center}
		\vskip0.7em}%
}

\date{}


\title{
	\centering
	\Large
	\textbf{LAPORAN KERJA PRAKTEK PROGRAM KAMPUS MERDEKA
		DI PT VOCASIA}
}

\subtitle{
	\centering
	\textbf{PERANCANGAN SISTEM POS (POINT OF SALES) UNTUK UMKM MENGGUNAKAN METODE WATERFALL BERBASIS WEBSITE}
}

\author{Oleh: \\ Maulana Sodiqin}

\begin{document}

\maketitle

\begin{center}
	\centering
	\textbf{
		PROGRAM STUDI TEKNIK INFORMATIKA
		FAKULTAS TEKNIK
		UNIVERSITAS ISLAM NUSANTARA
		BANDUNG
		2023/2024
	}
\end{center}

%\newpage

\section*{
  \small
  \centering
  BAB I : PENDAHULUAN
 }

\vspace{1.5pt}

\subsection*{
	\small
	1.1 Latar Belakang
}

\paragraph{}
Seiring dengan kemajuan teknologi informasi yang pesat, Usaha Mikro,
Kecil, dan Menengah (UMKM) sebagai salah satu pilar ekonomi nasional, turut
dihadapkan pada tuntutan untuk beradaptasi dengan perubahan zaman. Dalam
rangka mendukung perkembangan UMKM, PT Vocasia, sebagai perusahaan yang
berkomitmen untuk memberikan kontribusi positif pada sektor ini, merencanakan
rancangan dan implementasi Website Point of Sales (POS) sebagai solusi teknologi
untuk memperkuat manajemen penjualan dan operasional UMKM.

Dalam kaitannya dengan laporan kerja praktek ini, sebagai mahasiswa
Program Kampus Merdeka yang telah menempuh masa studi independen, merasa
perlu terlibat langsung dalam mengatasi tantangan yang dihadapi oleh UMKM
dalam mengadopsi teknologi, khususnya sistem POS berbasis web. Latar belakang
ini menjadi dasar utama penyelenggaraan Kerja Praktek ini, yang diarahkan untuk
mengidentifikasi, merancang, dan mengimplementasikan solusi yang relevan guna
meningkatkan efisiensi operasional UMKM.

Kerja Praktek ini diarahkan untuk memberikan kontribusi konkrit dalam
merancang dan mengimplementasikan Website POS yang tidak hanya efektif dan
efisien, tetapi juga mudah diadopsi oleh UMKM dengan berbagai tingkat keahlian
teknologi. Melalui laporan kerja praktek ini, diharapkan dapat ditemukan solusi
yang dapat membantu UMKM ,sambil memberikan pengalaman dan pengetahuan
yang mendalam di bidang pengembangan sistem informasi. Semua upaya ini
diarahkan untuk menciptakan ekosistem bisnis yang lebih berdaya saing, inovatif,
dan berkelanjutan untuk mendukung pertumbuhan ekonomi yang inklusif.



\newpage

\section*{
  \small
  \centering
  BAB I : PENDAHULUAN
 }

\vspace{1.5pt}

\subsection*{
	\small
	1.1 Latar Belakang
}

\paragraph{}
Seiring dengan kemajuan teknologi informasi yang pesat, Usaha Mikro,
Kecil, dan Menengah (UMKM) sebagai salah satu pilar ekonomi nasional, turut
dihadapkan pada tuntutan untuk beradaptasi dengan perubahan zaman. Dalam
rangka mendukung perkembangan UMKM, PT Vocasia, sebagai perusahaan yang
berkomitmen untuk memberikan kontribusi positif pada sektor ini, merencanakan
rancangan dan implementasi Website Point of Sales (POS) sebagai solusi teknologi
untuk memperkuat manajemen penjualan dan operasional UMKM.

Dalam kaitannya dengan laporan kerja praktek ini, sebagai mahasiswa
Program Kampus Merdeka yang telah menempuh masa studi independen, merasa
perlu terlibat langsung dalam mengatasi tantangan yang dihadapi oleh UMKM
dalam mengadopsi teknologi, khususnya sistem POS berbasis web. Latar belakang
ini menjadi dasar utama penyelenggaraan Kerja Praktek ini, yang diarahkan untuk
mengidentifikasi, merancang, dan mengimplementasikan solusi yang relevan guna
meningkatkan efisiensi operasional UMKM.

Kerja Praktek ini diarahkan untuk memberikan kontribusi konkrit dalam
merancang dan mengimplementasikan Website POS yang tidak hanya efektif dan
efisien, tetapi juga mudah diadopsi oleh UMKM dengan berbagai tingkat keahlian
teknologi. Melalui laporan kerja praktek ini, diharapkan dapat ditemukan solusi
yang dapat membantu UMKM ,sambil memberikan pengalaman dan pengetahuan
yang mendalam di bidang pengembangan sistem informasi. Semua upaya ini
diarahkan untuk menciptakan ekosistem bisnis yang lebih berdaya saing, inovatif,
dan berkelanjutan untuk mendukung pertumbuhan ekonomi yang inklusif.


\end{document}
